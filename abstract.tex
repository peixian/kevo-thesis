This study offers a novel exploration of the vital role of religious rituals among Karbala's devout populace, emphasizing the nuanced tensions between the laity and the clerical hierarchy. It scrutinizes various religious entities and ritual participants, including liturgy chanters, known as \emph{radūd}, the Iraqi militias called Hashd al-Shaabi, and local neighborhood assemblies, referred to as \emph{mūkib}. Drawing on unique ethnographic data collected in Karbala, this research reveals that lay Karbalaeis are not mere bystanders in their faith. They actively engage in an intricate contestation of the public sphere and significantly influence the evolution of their religious customs. Critically, this study diverges from the prevalent Najaf-centric research approach in Shi'a religious studies, where the focus has traditionally been on clerics and their relationships with the state or followers. It underscores the substantial role that grassroots actors in Karbala play in molding the religious terrain, thereby presenting a more comprehensive view of Iraq's religious environment that extends beyond Najaf's boundaries.