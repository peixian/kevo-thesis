What do religious rituals mean for pious residents of a shrine city? That is the central question I hope to answer in this thesis. Rituals themselves is already a leaky box of abstraction, intersecting with facets of lived experience, historical imperatives, and religious faith, yet rituals are also expected to play a key role in the formation of faith. 

For rituals to work, certain constraints must be placed upon them. They cannot be too loose with symbols and traditions, which risks tearing the ritual apart and losing meaning and followers. Yet too much rigidity poses the same threat of driving away followers, the nightmare of tradition cannot weigh too heavily on the laity. 

This thesis is focused on two main goals: an ethnographic description of the experience around rituals within and around Karbala, and an attempt to understand how these rituals provide space for new types of political discourse. The main finding of this thesis is, as a result of the massive expansion of outward performance in Shi'a piety in Iraq following 2003 and the Islamic State, authority within Shi'ism has become fractured into different groups. Old poles of power, such as tribes, have come into contact with new institutions and militias to compete for attention among the laity, making Shi'a rituals a reflection of the power struggles in Iraqi society. 

Some of the essential characteristics can be glimpsed from the unique political context of Karbala: a holy Shi'a city in rivalry with the other holy cities of Shi'ism and Islam in general, situated within a country crawling out of invasion and sectarian wars in the past two decades, all less than 20 kilometers from a cradle of civilization. The past presses close in Karbala, pulling its residents from the cacophony of the present and pushing them into becoming the agents they could otherwise never be. 

Having said what this thesis is about, I should also say what it is not about. The thesis is not a complete review of all Shi'a rituals, it is not the comparative nature of rituals across borders that I am interested in, but rather the merging of the holy city context with rituals. I am also unconcerned with the aspects of belonging and self-transformation that occur to pilgrims, the focus is on the institutions and people who make the rituals happen, as well as the residents of Karbala. 

Women are conspicuously absent within this thesis. This is an unfortunate facet of the Karbala context: as a man, nearly all rituals I observed was only men, and the conservative Islamic context prevented me from conducting extensive fieldwork with women. Rather than attempting to shoehorn a token section on the role of women, I hope that there is further study in the future by someone else on the role of women in Shi'a rituals. 

% As Geartz mentions in his seminal work on culture and religion: "the notion that religion tunes human actions to an envisaged cosmic order and projects images of cosmic order onto the plane of human experience is hardly novel. But it is hardly investigated either, so we have very little idea of how, in empirical terms, this particular miracle is accomplished." \cite[90]{geertz_interpretation_1993} While a pile of ethnographic work supports that this miracle is constructed through repeated actions, yet Geertz rejects this as a sufficient explanation, going on to mention that more theoretical works must be constructed. 

% Repetition is another oft studied facet. From Derrida's iteration and play to eternal pilgrims of Shikoku \cite{reader_pilgrims_2021}, pilgrimage, the idea that repetition itself creates space for change is also hardly novel. 

% What is the argument? The argument is that iteration, with language, photography, and iteration, is the key difference between this and seeing it as a pilgrimage. People aren't seen as pilgrims, especially in Karbala, because they're not pilgrims. Yet they still use the same transformative structures. 


\section{Background}
This project originally began as yet-another-study into sectarianism and sectarian identity formation within Iraq, specifically focused on the Hashd al-Shaabi. However after sustained fieldwork among the Hashd, my interests shifted into the banality of religious entwinement with the Hashd. After the sheen of militia studies had worn off as another bloody sword to swing, I realized that my research among the Hashd offered a gateway into peering into Karbala. 

Another motivating factor was that nearly every work on post-2003 Iraq centered around either poor people or politics. Due to a dearth of journalists and academics within Iraq, NGOs, especially IOM, had stepped in as local experts on the ground. This lead to journalists seeking NGOs as partners and "experts on the ground", creating a situation where NGOs effective became newsmakers, allowing NGOs to drive reporting and use media outlets as self justification. As NGOs primarily focus on poor people or politics, naturally work from Iraq centered on those two topics.

Even among NGOs, Karbala has remained a relatively understudied site. Najaf's close relations with the government, especially following recent law changes that requires Najaf to nominate the minister of religious endowments \cite{hamoudi_engagements_2020} has drawn the majority of focus towards Najaf. As a result, I decided that Karbala would prove to be a fruitful site in order to conduct research, although at the beginning my work still heavily focused on sectarianism and militia demobilization, disarmament, and reintegration. 

%My original background lies in Physics, Computer Science, and Linguistics. Natural language processing systems such as the GPT flavors have gradually taken over modern computer interfaces, with eschewing traditional linguistics studies of grammar morpology and formulic approaches to language. Frederick Jelinek, a researcher into automatic speech recognition, once famously said "Every time I fire a linguist, the performance of the speech recognizer goes up", referencing how sheer quantities of data have repeatedly outperformed brittle theories of language. I approached my work in the same way, in 

\section{Related Works}
The majority of research on Iraqi Shi'a rituals predates 2003, with a primary focus on historical accounts of rituals during the Ottoman era \cite{nakash_shiis_1994} \cite{brunner_twelver_2001} \cite{ayoub_redemptive_1978}. For instance, Yitzhak Nakash's work explores the origins of Shi'a religious rituals in the ninth century and emphasizes the evolution of the majlis tradition beyond the medieval period \cite[164]{yitzhak_nakash_attempt_1993}. Werner Ende, on the other hand, investigates prominent Shi'a ulema who participated in debates on the permissibility of self-flagellation and the various crises that ensued \cite{ende_flagellations_1978}. 

Post-2003 Shi'a religious rituals in Iraq have been scarcely documented, with the exception of Hamdan's 2012 master's thesis from the University of Arizona \cite{hamdan_development_2012}. In recent times, the research has mainly centered on Shi'a religious rituals within Iran \cite{dabashi_taziyeh_2005} \cite{aghaie_martyrs_2004} \cite{flaskerud_visualizing_2012} \cite{saramifar_circling_2020}, the Levant \cite{weiss_shadow_2010} \cite{szanto_beyond_2013}, or briefly touched upon Iraqi Shi'a rituals in the context of clerics and their connections to state power \cite{nakash_reaching_2006} \cite{davis_memories_2005} \cite{cole_ayatollahs_2006}. However, it is important to note that there are significant differences between Iranian and Iraqi Shi'a rituals, making Iranian literature only partially applicable to the Iraqi context.

Other Shi'a studies have focused on the context of Najaf, especially the \emph{hawza}, or Shi'a seminaries \cite{kassem_modernization_2018} \cite{mottahedeh_najaf_2016}. Other works on sectarianism, such as the work conducted by Fanar Haddad \cite{haddad_understanding_2020}, have focused on the political leadership of Najaf and the clerical class. As Mier Litvak has noted in his seminal work on the clerical differences between Karbala and Najaf in the nineteenth century \cite{litvak_shii_2002}, the two cities hold significant political differences that have filtered into daily life. This leaves a gap in understanding how Karbala fits into the religious politic. 

Nikkie Keddie's \emph{Religion and Politics in Iran} was the first to examine the political implications of Shi'a religious rituals, but her work is limited to Iran \cite{keddie_religion_1983}. Furthermore, the literature on the Levant primarily discusses the experiences of ritual participants in Lebanon or Syria and their role in pilgrimages to Damascus, which differs significantly from the political and spiritual context of Karbala. This results in a considerable gap in the literature on contemporary Iraqi rituals.

In terms of pilgrimage-related institutions, there has been significant research, such as Christopher Low's study on the precarious position of the Ottoman state in relation to water infrastructure in Mecca \cite{low_imperial_2020} and Lala Can's investigation of Sufi guest lodges as spiritual mediators between distant pilgrims and the Hajj \cite{can_spiritual_2020}. In East Asia, Wei-Ping Lin conducted an ethnography of pilgrimage on the demilitarized islands of Mazu, situated between China and Taiwan, analyzing the longing experienced by marginalized islanders \cite{lin_virtual_2014}. Reader and Shultz's ethnography on the Shikoku pilgrimage explores the role of temple infrastructure and various lodges, highlighting how public safety and access to free lodging provided by generous hotel owners along the pilgrimage route support pilgrims undertaking multiple journeys \cite{reader_pilgrims_2021}.

\section{Methodology}

Fieldwork was conducted in three major phases, two short visits of a month each in Karbala, a longer active fieldwork phase in Karbala, and a third smaller phase in comparative work in Baghdad. The first two phases of fieldwork were conducted in August 2021 and March 2022 respectively. The longer, active fieldwork consisted of fieldwork between May to October of 2022. A final phase of research focused on Baghdad from January to February 2023. 

Preliminary fieldwork largely was focused on building contacts and finding questions related to the fieldwork. In August 2021, my first visit was during Ashura in Karbala, with frequent travels to Najaf. Additional fieldwork was done in the Kurdish north, specifically with Shi'as residing in Duhok. March 2022 focused on Karbala specifically, with more in depth work about organization and institutions of Shi'a pilgrimage. Between May and October, I mostly resided in Karbala, but made frequent trips to other cities in order to do some comparative analysis, specifically the cities of Tel Afar, Samarra, Hilla, and Baghdad. In the two months of January and February of 2023, I decided to focus on fieldwork within Baghdad, since Baghdad's privileged role as the cultural and population center of Iraq has caused it to develop its own specific culture. 

The bulk of thesis relies on interviews conducted with Shi'a ritual practitioners, pilgrims, and everyday characters in Karbala. Other ethnographic accounts resulted from me attending and observing various rituals, as well as participating in pilgrimages to every Shi'a shrine within Iraq. 

\section{Organization}

This thesis is organized into two main content chapters. The rest of this introduction focuses on a description of Shi'a politics within contemporary Iraq, as well as a brief background of Shi'ism and religious rituals.

The first content chapter focuses on the role of institutions within Karbala, and how the institutions themselves shape rituals. Writing against traditional literature that focuses its role on the pilgrim, specifically with regards to tourism or the spiritual economy of pilgrimage, I attempt to show that the institutions play an active role in using pilgrims to further their own expansion, which also exposes them to change beyond their control. 

The second chapter focuses on a specific institution: the mowkeb. As the mowkeb lies at the heart of Iraqi Shi'a rituals, the second chapter draws out the history of mowkebs within Karbala, the specific sociolinguistic role they play, and how they act as the final piece in the bridge between tribes, politics, and religion. 

\section{Shiism and the Politics of Iraq}
Iraq's Shi'a majority city all hold specific pilgrimages, referred to as \emph{ziyara} (\begin{Arabic}زِيَارَة\end{Arabic}), or visitations. Four of them: Karbala, Najaf, Samarra, and Baghdad hold shrines for one or more of the twelve Imams. The Shrine of Imam Hussein (the third Imam) and the Shrine of Abbas are in Karbala, the Shrine of Imam Ali (the first Imam) lies in Najaf, the al-Askari Shrine in Samarra contains the tombs of Imam al-Hadi (the tenth Imam) and Imam al-Askari (the eleventh Imam), and the Shrine of Imam al-Kathim contains the tombs of Imam al-Kathim (the seventh Imam) and Imam al-Jawad (the nineth Imam). 

Each shrine has a bureaucratic institution called an \emph{ʿttaba} (\begin{Arabic}عتبة\end{Arabic}). Each \emph{ʿttaba} is responsible for soliciting donations, upkeep of the shrine, media relations, as well as other religious affairs. In Karbala specifically, the shrines of Abbas and Imam Hussein have separate \emph{ʿttabas}. Each \emph{ʿttaba} has an independent media affairs office, independent museums, radio stations, archives, and so on. The \emph{ʿttabas} are institutions of considerable size, funded by a combination of \emph{waqf} funds, state funds, and donations. In Karbala, the al-Kafeel foundation, the \emph{ʿttaba} for the shrine of Abbas, manages public health by running a series of public hospitals, including opening a new women's hospital in 2022. 

These bureaucratic expansions all largely occurred after 2003. During the Saddam years, \emph{ʿttaba} institutions were small and tightly monitored, limited to only maintenance of the shrines themselves \footnote{An interviewee described the \emph{ʿttabas} before 2003 as "shrine cleaning services", speaking to the limited role that was allowed during the Saddam era.}. Control and political power came after 2003, especially as article 43 of the 2005 constitution affirms the freedom of religions and sects to administer their \emph{waqf} affairs as they see fit. Article 43 even specifically mentions the freedom of ritual practice, including the Shi'a specific rituals, known as the \emph{Shʿir Husseiniyya} (\begin{Arabic}حسينية شعائر\end{Arabic}), or Husseini rituals \cite{jawad_iraqi_2003}. 

Haider Ala Hamoudi has mentioned how Article 43 has created an archipelago of diwans (administrative bodies for \emph{awqāf}) rather than a single uniform ministry \cite[220]{hamoudi_engagements_2020}. This relative freedom has combined with donation revenue from worldwide Shi'as to build massive bureaucracies to administer the shrines and expand the powers of the \emph{ʿttaba}. Sleek websites, public works projects, and media outreach has allowed the \emph{ʿttabas} of Abbas and Hussein to become the largest landowners in the old city of Karbala. 

Institutional rivalry is not only between specific \emph{ʿttabas}, but also between different cities as well. Karbala and Najaf have always held a historical rivalry, with Karbala being the center of pilgrimage, while Najaf remains as the Iraqi center of Shi'a studies in the form of \emph{hawza}. Although historically outside of the halls of political power in Baghdad, the emergence of a Shi'a dominated parliament in Baghdad has inexorably drawn Najafi clerics into politics. In recent years, this has manifested in the shape of the \emph{waqf} Bueau Law of 2012, which effectively gives veto power to the highest ranking Najafi cleric for the nominee of the head of the \emph{waqf} Bureau. 

Due to the centrality of \emph{waqf} within Islamic law, the rise of Shi'a \emph{awqāf} in south Iraq is indicative of merging of religion and politics. While politics and religion were artificially separated during the ostensibly secular Saddam era, Ayatollah Sistani, currently the highest ranking jurist in Najaf, played an active role in the formation of the 2005 Constitution \cite[146]{nakash_reaching_2006}. Sistani's 2014 "Wajib al-Kifai" calling for arms against the Islamic State lead to the formation of the Popular Mobilization Forces, also known as the Hashd al-Shaabi. From 2014 until the present, the Hashd evolved to become a paramilitary force that is loosely affiliated with the state, with their own media arm, medical centers, and functioning bureaucracy, all the while resisting calls to become further integrated into the Iraqi state. At the time of writing, the Hashd are almost akin to a parallel army within Iraq, including providing medical aid for foreign countries, such as Syria after the earthquakes in February 2023. Certain ties have developed between \emph{ʿttabas} and the Hashd as well. At any major shrine, pilgrims will wave Hashd flags, and donation boxes will be setup for injured Hashd veterans. The al-Kafeel foundation has a campus outside of Karbala used for combat medicine training as well. 

The linkage between \emph{awqāf}, clerics, and militias is to emphasize the point that the Shi'a rituals take place within this context. The explosion of Shi'a-affiliated bureaucracy after 2003 has resulted in pilgrims, especially pilgrim attention, being at the center. Institutions each vie for the pilgrims attention, billboards and radio stations announce the public works projects specific \emph{awqāf} or \emph{ʿttabas} do, the Hashd set up tents during pilgrimage season to provide medicine, food, and water to pilgrims, while also setting up small tent museums showcasing the body armor or weapons of Hashd martyrs. 

One tempting way of framing such actions is that the various institutions represent new loci of power emerging. Foreign aid money, combined with state revenues diverted by Shi'a politicians, and donation money combine to form powerful institutions, who require further substance and attention to grow. Individual clerics hold popular social media \cite{ann_wainscott_engaging_2019} for individual engagement as well. A laity hungry for Shi'a-affiliated content is now overwhelmed. Yet out of this arises a new type of engagement with institutions: because there are so many institutions, pilgrims now realize that their actions and the way they represent themselves in rituals carry meaning and power. This shift in power relies on repeated engagement between the religious institutions and the laity, expanding the scope of rituals beyond just a pilgrim and the deity. 

\section{Sha'ir Husseiniyya}
While the concept and scope of sha’ir husseiniyya or Husseini rituals has competing definitions, I will focus on the specific rituals conducted in Karbala during the month of Muharram and Safar, leading up to the 20th of Safar, which is the day of Arbaeen. 

The first two months of the Islamic calendar are Muharram and Safar. Within these months, there are two major dates for Shi’a Muslims, Ashura, the date when Imam Hussein was killed, and Arbaeen, the 40th day after the Battle of Karbala. While ‘Ashura represents the physical date of death (Ashura literally means 10), ‘Arbaeen’s history is a little more mixed. The most often agreed account was that 40 days after the death of Imam Hussein, one of the companions of the prophet, Jabir ibn. ‘Abd Allah al-Ansari visited the grave of Imam Hussein, and has been held to be the first pilgrim. For this reason, Arbaeen is seen as a pilgrimage date, while Ashura is seen as a mourning date. 

%Ashura (literally 10th) is the 10th day of Muharram, the first month in the Islamic Hijri calendar. Considered to be the day that Imam Hussein was martyred, the Shi'a believe that Hussein’s body remained undecomposed for three days, with the 13th of Muharram considered “burial day”. 

Muharram as a month is considered bad luck for Karbalaeis, being a whole month of mourning. In general, people tend not to make large purchases during Muharram such as furniture or cars, with the belief that such purchases are cursed and may catch fire. 

Logistically, the most amount of pilgrims arrive on Arbaeen, the 20th of Safar or 40 days after the death of Imam Hussein. While Ashura is still a large pilgrimage date, it is deemed acceptable to commemorate Ashura within one’s own city, with different Ashura celebrations in different cities of Iraq and across the world, including in New York and Najaf. While many Iraqis choose to travel to Karbala for Ashura, those with logistical difficulties will often celebrate it within their own cities, such as in northern cities such as Tel Afar, a Shi'a enclave within the Sunni north.  

Karbala has complicated logistics for pilgrimage, as the tradition of the mokweb must be regulated in order to accommodate the massive amount of pilgrims. Mowkebs are organizations which also set up tents to provide services for pilgrims, such as food, water, and lodging. This gives rise to a curious combination of state-run and clerical-run bureaucracies. Mowkebs within city limits are regulated by the Department of Mowkebs, which is run by the al-Kafeel foundation, affiliated with the Shrine of Abbas. The Department provides licenses to run mowkebs, which are then policed by the city. The Iraqi army and local police all provide security during non-peak seasons, but a few days before Ashura and Muharram additional security is provided by the Hashd al-Shaabi, a group of largely Shi'a militias that formed after the 2014 fatwa by Ayatollah Sistani, calling for action against the Islamic State. The Hashd themselves provide various mowkebs as well, in a combination of para-military, state, and religion. 

The Shrines of Imam Hussein and Abbas are centered in the old city of Karbala. Both shrines are massive, multistory mausoleums capable of holding thousands of people. Viewed from above, the two shrines form an oblong shaped, ringed by a single road and linked by large, tiled walkway known as \emph{bayn haramiiyn}\begin{Arabic}هرمين بين\end{Arabic}\footnote{Curiously, the same term of \emph{haramiyyn} is used by Sunnis and other Muslims to describe the holy cities of Mecca and Medina}. Bayn harmiin serves as a resting spot for pilgrims, with many families sitting in circles, as well as a site for rituals linking both shrines: one can walk into the Shrine of Abbas from the outer gate, into bayn harmiin, and into Shrine of Imam Hussein. 

The basic act of visitation to the shrine is called a \begin{Arabic}
    زيارة
\end{Arabic}\emph{ziyara} (lit. \emph{visit}). A pilgrim is expected to approach the shrine with a pure heart and clean clothes, and speak to the figure as if they were in the same room as them. The pilgrim first kisses the doors of the shrine similar to kissing in greeting a friend, then walks into the shrine to speak to the figure \cite{qisa_publications_illustrated_2018}. Within the center of the shrine is a the tomb room, coated in reflective crystal and glass. The tomb itself is encased within a large, barred cage, similar to a mortsafe. Pilgrims will often kiss the bars, or rub items against them in order to spiritually "bless" the items. 

Markets outside the shrines sell a variety of memorabilia and guides on visitation. Due to the mixture of backgrounds of the various pilgrims, a variety of visitation guides are sold in multiple languages, including Hindi, Urdu, Arabic, Farsi, and English. 

Shi'a religious rituals have faced various phases of discrimination and encouragement from state authorities for hundreds of years. Under Mamaluk control, Shi'a rituals were completely banned \cite{yitzhak_nakash_attempt_1993}. Even during Ottoman periods, policy towards pilgrimage and rituals varied between attempts of outright bans to encouragement, depending on the necessity of the loyalty of the Shi'a populace to the empire, as exemplified by the rule of Midhat Pasha \cite{aghaie_martyrs_2004}. More recently, during Saddam's rule, Shi'a rituals were banned and the Shrine of Imam Hussein was sacked in 1991 as a result of Shi'a uprisings after the Gulf War. Pilgrimage was outright banned after 1991, interviewees described how they would walk in circuitous routes between different villages in order to perform pilgrimage between Najaf and Karbala. After 2003, a constitutional article specifically mentions freedom to practice Shi'a rituals, leading to a massive growth in participation and an explosion of Shi'a affiliated institutions and bureaucracies. 

This thesis commences by examining the post-2003 and post-ISIS context in which the emerging Shi'a bureaucracy has converged with the historical imperative of a holy city, transforming Karbala into a battleground. The lived religion of Karbalaeis is now infused with complex meanings, as they find themselves caught between competing powers of the Iraqi state, militias, the city, and tribes. This study aims to analyze rituals and pilgrimage beyond personal experiences, viewing them as open-ended processes that shape a specific future.

Rather than simple narratives which emphasize the role of clerics in Najaf, we see that Karbala exists as counter-pole to clerical control. I argue that Shi'a Husseini rituals serve as the primary platform in molding lived Shiism in Karbala.  Each actor, whether consciously or not, comprehends their role in this intricate dynamic. Through this lens, the experiences of Karbalaeis provide fresh insight into how rituals intersect with the various centers of power, and how the laity can derive their influence from these rituals. 