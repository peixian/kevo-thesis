% What does Karbala mean for pious residents?

What do religious rituals mean for pious residents of a shrine city? That is the central question I hope to answer in this thesis. Rituals themselves is already a leaky box of abstraction, intersecting with facets of lived experience, historical imperatives, and religious faith, crossing the currents between the laity and the clerics.  

% Clifford Geertz famously identified the role of symbols in religion and the tensions these symbols represent \cite{geertz_interpretation_1993}. Rituals as part of religion functions in the same way and holds similar tensions, rituals cannot be too loose with symbols and traditions, which risks tearing the ritual apart and losing meaning and followers. Yet too much rigidity poses the same threat of driving away followers, the nightmare of tradition cannot weigh too heavily on the laity. The "aura of factuality", as Geertz mentions, is the fine line that rituals must straddle in order to maintain popular appeal \cite[90]{geertz_interpretation_1993}. 

This thesis is focused on two main goals: an original ethnographic account of the experience around rituals and their institutions within Karbala, and providing a different frame of analysis for understanding political life within south Iraq. The majority of studies on Shi'ism in Iraq emphasize the role of Najaf and clerical authority and treat religious authority as directly translatable into political authority. This elision in the cleric-centric framing helps us only understand one aspect of political life and obscures other, additional frames. 

The main finding of this thesis is that after 2003 and 2014 with the fall of Saddam Hussein's regime and the defeat of the Islamic State (ISIS), authority within the Shi'a community in Iraq has fractured between the various shrine  bureaucracies, the leaders of rituals, and the clerical class. This is in addition to the ever-present history of tribal culture in Iraq, which exerts strong influences over community organization in Karbala today. Rituals, in this sense, are both a reflection and a refraction of Iraqi society: they reflect the growing divisions of power, but also refract and bend the public views, allowing Karbala residents to articulate their grievances. Through an ethnography of the various ritual institutions of Karbala, I show that cleric authority has been eroded in favor of alternative forms of authority. 

% Some of the essential characteristics can be glimpsed from the unique political context of Karbala: a holy Shi'a city in rivalry with the other holy cities of Shi'ism and Islam in general. The past presses close in Karbala, pulling its residents from the cacophony of the present and pushing them into becoming the agents they could otherwise never be. 

Having said what this thesis is about, I should also say what it is not about. It is not a complete cataloging of Shi'a rituals in Karbala\footnote{For a much more complete catalog, see \cite{hamdan_development_2012}.}. It is also not a direct account of the role of the clerical class, as it shifts the spotlight to the lay constituency. I am also unconcerned with the aspects of belonging and self-transformation that occur to pilgrims, shifting the focus methodologically is on the institutions and people who make the rituals happen within Karbala.

Women are conspicuously absent within this thesis. This is an unfortunate facet of research in Karbala: as a man, nearly all rituals I observed involved only men, and the conservative Islamic context prevented me from conducting extensive fieldwork with women. Rather than attempting to shoehorn a token section on the role of women, I hope that there is further study in the future by someone else on the role of women in Shi'a rituals\footnote{Edith Szanto's work\cite{szanto_beyond_2013} has already mentioned how women encourage and act as audiences in rituals in Damascus, and I believe a similar study can be done in Karbala.}. 

% As Geartz mentions in his seminal work on culture and religion: "the notion that religion tunes human actions to an envisaged cosmic order and projects images of cosmic order onto the plane of human experience is hardly novel. But it is hardly investigated either, so we have very little idea of how, in empirical terms, this particular miracle is accomplished." \cite[90]{geertz_interpretation_1993} While a pile of ethnographic work supports that this miracle is constructed through repeated actions, yet Geertz rejects this as a sufficient explanation, going on to mention that more theoretical works must be constructed. 

% Repetition is another oft studied facet. From Derrida's iteration and play to eternal pilgrims of Shikoku \cite{reader_pilgrims_2021}, pilgrimage, the idea that repetition itself creates space for change is also hardly novel. 

% What is the argument? The argument is that iteration, with language, photography, and iteration, is the key difference between this and seeing it as a pilgrimage. People aren't seen as pilgrims, especially in Karbala, because they're not pilgrims. Yet they still use the same transformative structures. 


\section{Background}
This project originally began as yet-another-study into sectarianism and sectarian identity formation within Iraq, specifically focused on the Hashd al-Shaabi, the militia groups which arose after the fall of Mosul to ISIS in response to a fatwa from Ayatollah Sistani in 2014 \cite{jonathan_stevenson_shia_2017}. However after sustained fieldwork among the Hashd, my interests shifted into the banality of religious entwinement with the Hashd. After the sheen of militia studies had worn off as another bloody sword to swing, I realized that my research among the Hashd offered a gateway into peering into Karbala. 

% Another motivating factor was that nearly every work on post-2003 Iraq centered around either poor people or politics. Due to a dearth of journalists and academics within Iraq, NGOs, especially IOM, had stepped in as local experts on the ground. This lead to journalists seeking NGOs as partners and "experts on the ground", creating a situation where NGOs effective became newsmakers, allowing NGOs to drive reporting and use media outlets as self justification. As NGOs primarily focus on poor people or politics, naturally work from Iraq centered on those two topics.

Karbala has remained a relatively understudied in academia. After conducting initial fieldwork in Najaf, I realized that both Najaf and Karbala were institutional and political rivals. Najaf's close relations with the government, especially following recent law changes that requires Najaf to nominate the minister of religious endowments \cite{hamoudi_engagements_2020} has drawn the majority of focus. In addition, Najafi Shi'a studies is overshadowed by the main Shi'a seminary, Karbala has no such institution that draws clerical attention. I decided that Karbala as an understudied site would provide a broader view of understanding the role of religious rituals for the average Shi'a Iraqi. 

%My original background lies in Physics, Computer Science, and Linguistics. Natural language processing systems such as the GPT flavors have gradually taken over modern computer interfaces, with eschewing traditional linguistics studies of grammar morpology and formulic approaches to language. Frederick Jelinek, a researcher into automatic speech recognition, once famously said "Every time I fire a linguist, the performance of the speech recognizer goes up", referencing how sheer quantities of data have repeatedly outperformed brittle theories of language. I approached my work in the same way, in 

\section{Related Works}
The majority of research on Iraqi Shi'a rituals predates 2003, with a primary focus on historical accounts of rituals during the Ottoman era \cite{nakash_shiis_1994} \cite{brunner_twelver_2001} \cite{ayoub_redemptive_1978}. For instance, Yitzhak Nakash's work explores the origins of Shi'a religious rituals in the ninth century and charts the post-medieval evolution of a Shi'a ritual known as the majlis\footnote{See section \ref{majlis} for a detailed description.} \cite[164]{yitzhak_nakash_attempt_1993}. Werner Ende, on the other hand, investigates prominent Shi'a ulema who participated in debates on the permissibility of self-flagellation and the various crises that ensued \cite{ende_flagellations_1978}. 

Post-2003 Shi'a religious rituals in Iraq have been scarcely documented, with the exception of Hamdan's 2012 master's thesis from the University of Arizona \cite{hamdan_development_2012}. In recent times, the research has mainly centered on Shi'a religious rituals within Iran \cite{dabashi_taziyeh_2005} \cite{aghaie_martyrs_2004} \cite{flaskerud_visualizing_2012} \cite{saramifar_circling_2020}, the Levant \cite{weiss_shadow_2010} \cite{szanto_beyond_2013}, or briefly touched upon Iraqi Shi'a rituals in the context of clerics and their connections to state power \cite{nakash_reaching_2006} \cite{davis_memories_2005} \cite{cole_ayatollahs_2006}. However, it is important to note that there are significant differences between Iranian and Iraqi Shi'a rituals, making Iranian literature only partially applicable to the Iraqi context.

Other Shi'a studies have focused on the context of Najaf, especially the \emph{hawze}, or Shi'a seminaries \cite{kassem_modernization_2018} \cite{mottahedeh_najaf_2016}. Other works on sectarianism, such as the work conducted by Fanar Haddad \cite{haddad_understanding_2020}, have focused on the political leadership of Najaf and the clerical class. As Mier Litvak has noted in his seminal work on the clerical differences between Karbala and Najaf in the nineteenth century \cite{litvak_shii_2002}, the provincial small town nature of Karbala and Najaf have set the communities apart from the metropolitan Sunni clerical classes based in Istanbul, Cairo, or Damascus \cite[3]{litvak_shii_2002}. However, Najaf's monopolization of leadership \cite[64]{litvak_shii_2002}, including its role in drawing the most famous \emph{hawze}, has created two different constituencies between the cities. This monopolization has subsequently caused most studies to be focused on Najaf, which leaves a gap in understanding Karbala.

Nikkie Keddie's \emph{Religion and Politics in Iran} was the first English work to examine the political implications of Shi'a religious rituals, but her work is limited to Iran \cite{keddie_religion_1983}. Furthermore, the literature on the Levant primarily discusses the experiences of ritual participants in Lebanon or Syria and their role in pilgrimages to Damascus, which differs significantly from the political and spiritual context of Karbala. This results in a considerable gap in the literature on contemporary Iraqi rituals.

In terms of pilgrimage-related institutions, there has been significant research, such as Christopher Low's study on the precarious position of the Ottoman state in relation to water infrastructure in Mecca \cite{low_imperial_2020} and Lala Can's investigation of Sufi guest lodges as spiritual mediators between distant pilgrims and the Hajj \cite{can_spiritual_2020}. In East Asia, Reader and Shultz's ethnography on the Shikoku pilgrimage explores the role of temple infrastructure and various lodges, highlighting how public safety and access to free lodging provided by generous hotel owners along the pilgrimage route support pilgrims undertaking multiple journeys \cite{reader_pilgrims_2021}.

Jürgen Habermas developed the idea of "public sphere theory", which is concerned with the spaces and mechanisms where public opinion is created and political discussions are shaped \cite{habermas_structural_1999}. Scholars like James Holston \cite{holston_insurgent_2008} and Arjun Appadurai \cite{appadurai_fear_2006} have built upon public sphere theory, emphasizing the importance of understanding how state institutions and policies influence the dynamics of the public sphere, particularly in non-Western and postcolonial contexts. Michael Warner's work on publics and counterpublics also lend a background in understanding how different publics are shaped, specifically in his idea that mere attention qualifies as a public, and this qualification only requires active uptake \cite[87]{warner_publics_2010}. Holsten's work on marginalized urban residents in São Paulo describes the process of insurgent citizenship, where the public sphere is transformed through the activism of marginalized urban residents who mobilized along community and neighborhood lines. This work is particularly helpful in understanding Karbala, as the tribal influences in Karbala have created distinct neighborhood identities that are activated while performing religious rituals.

Saba Mahmood's work on Salafi women challenged the assumption that agency can only be expressed through resistance to norms or power structures, showing how religious rituals may be used in people's willingness to engage with and shape their practices \cite{mahmood_politics_2005}. Similarly, Focault's work on "technologies of the self" describes how subjects shape their own subjectivties, moral conduct, and self-understanding \cite{foucault_technologies_1988}.

% Other works on Karbala are focused on the role of the disapora or Shi'a residing in different contexts, such as India \cite{pinault_horse_2016} or London \cite{dogra_karbala_2017} \cite{marei_lamenting_2021}. While contributing greatly to understanding how Shi'a reinterpret rituals within a different context, these studies are nevertheless significantly different than a study conducted physically in Karbala, as they focus on the interaction of a minority community within a completely different context. 

This thesis builds upon the preexisting works of Shi'a religious rituals and Karbala by providing the Iraqi perspective. It extends the research done by Keddie's work and borrows methodological tools from Low, Can, and Reader by shifting the framing to the institutions that support religious rituals and pilgrimage, rather than the pilgrims themselves. Public sphere theory, although operating on radically different scales, provides useful tools to understand rituals, as the rituals create and alter different publics through coming into the range of contact with others experiencing and directing attention. When combined with Mahmood's and Focault's ideas on the self to show that viewing Iraqi Shi'ism through a cleric and Najaf focused frame obscure political changes that rituals create and reflect.

\section{Methodology}
Fieldwork was conducted in three major phases, two short visits of a month each in Karbala, a longer active fieldwork phase in Karbala, and a third smaller phase in comparative work in Baghdad. The first two phases of fieldwork were conducted in August 2021 and March 2022 respectively. The longer, active fieldwork consisted of fieldwork between May to October of 2022. A final phase of research focused on Baghdad from January to February 2023. 

Preliminary fieldwork largely was focused on building contacts and finding questions related to the fieldwork. In August 2021, my first visit was during Ashura within Karbala and Najaf. Additional fieldwork was done in the Kurdish north within this time, specifically with Shi'as residing in Duhok. March 2022 focused on Karbala specifically, with more in depth work about organization and institutions of Shi'a pilgrimage. Between May and October, I mostly resided in Karbala conducting interviews and attending ritual ceremonies, but made frequent trips to the cities of Tel Afar, Samarra, Hilla, and Baghdad in order to find out if there were differences in rituals. In the two months of January and February of 2023, I decided to focus on fieldwork within Baghdad, since Baghdad's privileged role as the cultural and population center of Iraq has caused it to develop its own specific culture. 

The bulk of thesis relies on interviews and observations conducted with Shi'a ritual practitioners, pilgrims, and everyday characters in Karbala. For interviews with workers at the shrines, two different types of interviewers were conducted: official interviews that were recorded with multiple people in attendance, and informal interviews conducted with workers without official written blessing. This distinction was necessary because after a BBC documentary on sex trafficking within Karbala in 2019 \footnote{See https://www.bbc.co.uk/mediacentre/latestnews/2019/undercover-with-the-cleric.}, the shrine has clamped down on recording devices and notes. On many occasions I was aggressively asked about my voice recorder and notebook. Many ordinary workers would only talk to me after I assured them that I was not a journalist and also will not publish their names. As a result, only the official interviews are cited with the interviewee's name, the informal interviews are cited with a location and date. 

Other ethnographic accounts resulted from me attending and observing various rituals, as well as participating in pilgrimages to every Shi'a shrine within Iraq. My settings for interlocutors ranged from shrines and mosques to coffee shops and libraries, typically in informal conversations without a translator. Interviewees varied from regular pilgrims to residents of Karbala, as well as government officials and members of the Iraqi security forces. In addition, I conducted additional interviews while walking the Arbaeen pilgrimage from Najaf to Karbala in September of 2022, as well as the pilgrimage to the Shrine of Imam al-Kazim in February 2023. While most rituals were recorded in video or audio, the majority of interviews were only recorded via written notes, as to preserve a casual atmosphere with the interviewee. 

Some archival work was conducted within the Institute for Shi'a Studies in Karbala, the Karbala Central Library, and the Iraqi National Archives in Baghdad. However, as I conducted additional fieldwork, I realized that the historical archive of Shi'a religious rituals would be too expansive to cover within a single project, leaving me to set aside much of the historical work. Archival work also became parallel to ethnographic work, as I realized many of these rituals were bound as oral traditions, and the bulk of the history was never stored in archives. 

\section{Organization}

This thesis is organized into two main content chapters. The rest of this introduction focuses on a description of Shi'a politics within contemporary Iraq, as well as a brief background of Shi'ism and religious rituals.

Chapter \ref{chapter-1} focuses on the role of institutions within Karbala, and how the institutions themselves shape rituals. Writing against literature that focuses on the role of Najafi clerics, I show that alternative sources of religious authority are springing up in the years after ISIS.

Chapter \ref{chapter-2} focuses on a specific institution: the \emph{mūkeb}\footnote{Singular \emph{mūkeb}, plural \emph{muwālkib}}. The \emph{mūkeb} is a bureaucratic organization structure that lies at the heart of Iraqi Shi'a rituals. This chapter draws out the history of \emph{mūkeb} within Karbala, showing the specific role they play within Karbala, and discusses a religious ritual that is native to them, called the \emph{radat}.

Throughout this thesis, the Arabic is plural is used for plurals instead of the English plural of the Arabic word in order to preserve clarity on the topic. Per IJMES transliteration guidelines rules 3, 4, and 5, only technical terms contain diacritics: general words found in the dictionary and names of prominent figures and titles are presented without diacritics. Words such as "Imam" and names of specific figures words are left without diacritics and use the widely accepted transliteration without italics, while italics are reserved for technical terms.

\section{Shi'ism and the Politics of Contemporary Iraq}
Iraq's Shi'a majority cities all hold specific pilgrimages, referred to as \emph{ziyara}, or visitations. Four of them: Karbala, Najaf, Samarra, and Baghdad hold shrines for one or more of the twelve Imams. The Shrine of Imam Hussein (the third Imam) and the Shrine of Abbas are in Karbala, the Shrine of Imam Ali (the first Imam) lies in Najaf, the al-Askari Shrine in Samarra contains the tombs of Imam al-Hadi (the tenth Imam) and Imam al-Askari (the eleventh Imam), and the Shrine of Imam al-Kadhim contains the tombs of Imam al-Kadhim (the seventh Imam) and Imam al-Jawad (the nineth Imam). 

Each shrine has a bureaucratic institution called an \emph{ʿtaba}\footnote{Singular \emph{ʿtaba}, plural \emph{ʿtabat}}. Each \emph{ʿtaba} is responsible for soliciting donations, upkeep of the shrine, media relations, as well as other religious affairs. In Karbala specifically, the shrines of Abbas and Imam Hussein have separate \emph{ʿtabat}. Each \emph{ʿttaba} has an independent media affairs office, independent museums, radio stations, archives, and so on. The \emph{ʿtabat} are institutions of considerable size, funded by a combination of \emph{waqf}\footnote{Singular \emph{waqf}, plural \emph{awqāf}} funds, state funds, and donations. In Karbala, the al-Kafeel foundation, the \emph{ʿtaba} for the shrine of Abbas, manages public health by running a series of public hospitals, including a new women's hospital open in 2022. 

These bureaucratic expansions all largely occurred after 2003. During the Saddam years, \emph{ʿtaba} institutions were small and tightly monitored, limited to only maintenance of the shrines themselves \footnote{An interviewee described the \emph{ʿtabat} before 2003 as "shrine cleaning services", speaking to the limited role that was allowed during the Saddam era.}. Control and political power came after 2003, especially as Article 43 of the 2005 constitution affirms the freedom of religions and sects to administer their \emph{waqf} affairs as they see fit, including the Shi'a specific rituals, known as the \emph{Shaʿir Husseiniyya}, or Husseini rituals \cite{jawad_iraqi_2003}. 

Haider Ala Hamoudi has mentioned how Article 43 has created an archipelago of diwans (administrative bodies for \emph{awqāf}) rather than a single uniform ministry \cite[220]{hamoudi_engagements_2020}. This relative freedom has combined with donation revenue from worldwide Shi'as to build massive bureaucracies to administer the shrines and expand the powers of the \emph{ʿtaba}. Sleek websites, public works projects, and media outreach has allowed the \emph{ʿtabat} of Abbas and Hussein to become the largest landowners in the old city of Karbala. 

Institutional rivalry is not only between specific \emph{ʿtabat}, but also between different cities as well. Karbala and Najaf have always held a historical rivalry, with Karbala being the center of pilgrimage, while Najaf remains as the Iraqi center of Shi'a studies in the form of religious seminaries, known as \emph{hawza}. Although historically outside of the halls of political power in Baghdad, the emergence of a Shi'a dominated parliament in Baghdad has inexorably drawn Najafi clerics into politics. In recent years, this has manifested in the shape of the \emph{waqf} Bueau Law of 2012, which effectively gives veto power to the highest ranking Najafi cleric for the nominee of the head of the \emph{waqf} Bureau \cite{hamoudi_engagements_2020}.

% Due to the centrality of \emph{waqf} within Islamic law, the demise of the \emph{waqf} in Sunni states has mirrored the rise of \emph{awqāf} in Shi'a countries \cite{hamoudi_engagements_2020}. The 

While politics and religion were artificially separated during the ostensibly secular Saddam era, Ayatollah Sistani, currently the highest ranking jurist in Najaf, played an active role in the formation of the 2005 Constitution \cite[146]{nakash_reaching_2006}. Sistani's 2014 fatwa, \emph{Wajib al-Kifai}, called for arms against the Islamic State and lead to the formation of the Popular Mobilization Forces, also known as the Hashd al-Shaabi. From 2014 until the present, the Hashd evolved to become a paramilitary force that is loosely affiliated with the state, with their own media arm, medical centers, and functioning bureaucracy, all the while resisting calls to become further integrated into the Iraqi state. At the time of writing, the Hashd are almost akin to a parallel army within Iraq, including providing medical aid for foreign countries, such as Syria after the earthquakes in February 2023. Certain ties have developed between \emph{ʿtabat} and the Hashd as well; at any major shrine, pilgrims will wave Hashd flags, and donation boxes will be setup for injured Hashd veterans. The al-Kafeel foundation has a campus outside of Karbala used for combat medicine training as well. 

The linkage between \emph{awqāf}, clerics, and militias is to emphasize the point that the Shi'a rituals take place within this context. The explosion of Shi'a-affiliated bureaucracy after 2003 has resulted in new efforts to draw in pilgrims and pilgrim attention. Institutions each vie for the pilgrims attention, billboards and radio stations announce the public works projects specific \emph{awqāf} or \emph{ʿtabat} do, the Hashd set up service tents during pilgrimage season to provide medicine, food, and water to pilgrims, while also setting up small tent museums showcasing the body armor or weapons of Hashd martyrs. 

Cleric-centric framings around Shi'ism occlude these lived practices, overlooking that the shrines are public spheres where multiple authorities swing between contestation and cooperation. Examinations of the Hashd purely as militias and violence also overlooks the complex mechanics at work, especially with relations to bureaucracy. More interestingly, Karbalaeis are completely at peace with this tension, often times finding that there is no tension in this continually shifting sphere.

% One tempting way of framing such actions is that the various institutions represent new loci of power emerging. Foreign aid money, combined with state revenues diverted by Shi'a politicians, and donation money combine to form powerful institutions, who require further substance and attention to grow. Individual clerics hold popular social media \cite{ann_wainscott_engaging_2019} for individual engagement as well. A laity hungry for Shi'a-affiliated content is now overwhelmed. Because there are so many institutions, pilgrims now realize that their actions and the way they represent themselves in rituals carry meaning and power. Pilgrim attention has become a vital currency to the lifebood of these institutions, relying on repeated engagement between the religious institutions and the laity. 

\section{\emph{Shʿir Husseiniyya}}
While the concept and scope of \emph{shʿir husseiniyya} or Husseini rituals has competing definitions, I will focus on the specific rituals conducted in Karbala during the month of Muharram and Safar, leading up to the 20th of Safar, which is the day of Arbaeen. 

The first two months of the Islamic calendar are Muharram and Safar. Within these months, there are two major dates for Shi’a Muslims, Ashura, the date when Imam Hussein was killed, and Arbaeen, the 40th day after the Battle of Karbala. While Ashura represents the physical date of death\footnote{Ashura literally means 10.}, Arbaeen’s history is a little more mixed. The most often agreed account was that 40 days after the death of Imam Hussein, one of the companions of the prophet, Jabir ibn. ‘Abd Allah al-Ansari visited the grave of Imam Hussein, and has been held to be the first pilgrim. For this reason, Arbaeen is seen as a pilgrimage date, while Ashura is seen as a mourning date. 

%Ashura (literally 10th) is the 10th day of Muharram, the first month in the Islamic Hijri calendar. Considered to be the day that Imam Hussein was martyred, the Shi'a believe that Hussein’s body remained undecomposed for three days, with the 13th of Muharram considered “burial day”. 

Muharram as a month is considered bad luck for Shi'as and Sunni Muslims who venerate the family of Imam Ali, being a whole month of mourning. In Karbala, residents tend not to make large purchases during Muharram such as furniture or cars, with the belief that such purchases are cursed and may catch fire. 

Logistically, the most amount of pilgrims arrive on Arbaeen, the 20th of Safar or 40 days after the death of Imam Hussein. While Ashura is draws large crowds of pilgrims, they are mostly Iraqis who can travel to Karbala, as it is deemed acceptable to commemorate Ashura within one’s own city, with different Ashura celebrations in different cities of Iraq and across the world, including in New York and Najaf. While many Iraqis choose to travel to Karbala for Ashura, those with logistical difficulties will often celebrate it within their own cities, such as in northern cities such as Tel Afar, a Shi'a enclave within the Sunni north.  

Karbala has complicated logistics for pilgrimage, as the tradition of the \emph{mūkeb} must be regulated in order to accommodate the massive amount of pilgrims. \emph{Muwālkib}  are organizations which also set up tents to provide services for pilgrims, such as food, water, and lodging. This gives rise to a curious combination of state-run, volunteer-run, and clerical-run bureaucracies. \emph{Muwālkib}  within city limits are regulated by the Department of \emph{Muwālkib} , which is run by the al-Kafeel foundation, affiliated with the Shrine of Abbas. The Department provides licenses to run \emph{muwālkib} , which are then policed by the city. The Iraqi army and local police all provide security during non-peak seasons, but a few days before Ashura and Muharram additional security is provided by the Hashd al-Shaabi. The Hashd themselves provide various services, including food and water.

The Shrines of Imam Hussein and Abbas are centered in the old city of Karbala. Both shrines are massive, multistory mausoleums capable of holding thousands of people. Viewed from above, the two shrines form an oblong shaped, ringed by a single road and linked by large, tiled walkway known as \emph{bayn haramayn}\footnote{Curiously, the same term of \emph{haramīn} is used by Sunnis and other Muslims to describe the holy cities of Mecca and Medina.}. \emph{Bayn haramīn} serves as a resting spot for pilgrims, with many families sitting in circles, as well as a site for rituals linking both shrines: one can walk into the Shrine of Abbas from the outer gate, into \emph{bayn haramīn}, and into Shrine of Imam Hussein. 

The basic act of visitation to the shrine is called a \emph{ziyara}. A pilgrim is expected to approach the shrine with a pure heart and clean clothes, and speak to the figure as if they were in the same room as them. The pilgrim first kisses the doors of the shrine similar to kissing in greeting a friend, then walks into the shrine to speak to the figure \cite{qisa_publications_illustrated_2018}. Within the center of the shrine is a the tomb room, coated in reflective crystal and glass. The tomb itself is encased within a large, barred cage, similar to a mortsafe. Pilgrims will often kiss the bars, or rub items against them in order to spiritually "bless" the items. 

Markets outside the shrines sell a variety of memorabilia and guides on visitation. Shi'a prayer stones are sold alongside carved stones, jewelery, and black t-shirts that represent a pilgrim's visitation to Karbala. The hawkers are native Karbalaeis, usually descended from a family who has been running the shops for many years. Due to the mixture of backgrounds of the various pilgrims, a variety of visitation guides are sold in multiple languages, including Hindi, Urdu, Arabic, Farsi, and English. 

Shi'a religious rituals have faced various phases of discrimination and encouragement from state authorities for hundreds of years. Under Mamaluk control, Shi'a rituals were completely banned \cite{yitzhak_nakash_attempt_1993}. Even during Ottoman periods, policy towards pilgrimage and rituals varied between attempts of outright bans to encouragement, depending on the necessity of the loyalty of the Shi'a populace to the empire, as exemplified by the rule of Midhat Pasha \cite{aghaie_martyrs_2004}. More recently, during Saddam's rule, the government banned Shi'a rituals and sacked the Shrine of Imam Hussein in response to the Shi'a uprisings after the Gulf War. Pilgrimage was outright banned after 1991. Interviewees described how they would walk in circuitous routes between different villages in order to perform pilgrimage between Najaf and Karbala. The 2005 constitutional specifically guarantees the freedom to practice Shi'a rituals, which has lead to a massive growth in participation and an explosion of Shi'a affiliated institutions and bureaucracies. 

This thesis starts by examining the post-2003 and post-ISIS context in which the emerging Shi'a organizations has converged with the historical imperative of a holy city, transforming Karbala into multiple publics tied together by religious rituals. The Hashd, ritual leaders, and organizations that manage the shrines all demand attention, creating separate publics that merge together into the same physical space. 